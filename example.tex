% This example is meant to be compiled with lualatex or xelatex
% The theme itself also supports pdflatex
\PassOptionsToPackage{unicode}{hyperref}
\documentclass[aspectratio=1610, 9pt]{beamer}

% Warning, if another latex run is needed
% \usepackage[aux]{rerunfilecheck}

% just list chapters and sections in the toc, not subsections or smaller
\setcounter{tocdepth}{1}

%------------------------------------------------------------------------------
%------------------------------ Fonts, Unicode, Language ----------------------
%------------------------------------------------------------------------------
\usepackage{fontspec}
\defaultfontfeatures{Ligatures=TeX}  % -- becomes en-dash etc.

% german language
\usepackage{polyglossia}
\setdefaultlanguage{german}

% for english abstract and english titles in the toc
\setotherlanguages{english}

% intelligent quotation marks, language and nesting sensitive
\usepackage[autostyle]{csquotes}

% microtypographical features, makes the text look nicer on the small scale
\usepackage{microtype}

%------------------------------------------------------------------------------
%------------------------ Math Packages and settings --------------------------
%------------------------------------------------------------------------------

\usepackage{amsmath}
\usepackage{amssymb}
\usepackage{mathtools}
\usepackage{bbold}

% Enable Unicode-Math and follow the ISO-Standards for typesetting math
\usepackage[
  math-style=ISO,
  bold-style=ISO,
  sans-style=italic,
  nabla=upright,
  partial=upright,
]{unicode-math}
\setmathfont{Latin Modern Math}

% nice, small fracs for the text with \sfrac{}{}
\usepackage{xfrac}


%------------------------------------------------------------------------------
%---------------------------- Numbers and Units -------------------------------
%------------------------------------------------------------------------------

\usepackage[
  locale=DE,
  separate-uncertainty=true,
  per-mode=symbol-or-fraction,
]{siunitx}
\sisetup{math-micro=\text{µ},text-micro=µ}
% \sisetup{tophrase={{ to }}}
%------------------------------------------------------------------------------
%-------------------------------- tables  -------------------------------------
%------------------------------------------------------------------------------

\usepackage{booktabs}       % \toprule, \midrule, \bottomrule, etc

%------------------------------------------------------------------------------
%-------------------------------- graphics -------------------------------------
%------------------------------------------------------------------------------

\usepackage{graphicx}
%\usepackage{rotating}
\usepackage{grffile}
\usepackage{tikz}
\usepackage{circuitikz}
\usepackage{tikz-feynman}
\usepackage{subcaption}

% allow figures to be placed in the running text by default:
\usepackage{scrhack}
\usepackage{float}
\floatplacement{figure}{htbp}
\floatplacement{table}{htbp}

% keep figures and tables in the section
\usepackage[section, below]{placeins}


%------------------------------------------------------------------------------
%---------------------- customize list environments ---------------------------
%------------------------------------------------------------------------------

\usepackage{enumitem}
\usepackage{listings}
\usepackage{hepunits}

\usepackage{pdfpages}
%------------------------------------------------------------------------------
%------------------------------ Bibliographie ---------------------------------
%------------------------------------------------------------------------------

\usepackage[
  backend=biber,   % use modern biber backend
  autolang=hyphen, % load hyphenation rules for if language of bibentry is not
                   % german, has to be loaded with \setotherlanguages
                   % in the references.bib use langid={en} for english sources
]{biblatex}
\addbibresource{references.bib}  % the bib file to use
\DefineBibliographyStrings{german}{andothers = {{et\,al\adddot}}}  % replace u.a. with et al.


% Load packages you need here
% \usepackage{polyglossia}
% \setmainlanguage{german}

\usepackage{csquotes}


% \usepackage{amsmath}
% \usepackage{amssymb}
% \usepackage{mathtools}

\usepackage{hyperref}
\usepackage{bookmark}

% load the theme after all packages

\usetheme[
  showtotalframes, % show total number of frames in the footline
]{tudo}

% Put settings here, like
\unimathsetup{
  math-style=ISO,
  bold-style=ISO,
  nabla=upright,
  partial=upright,
  mathrm=sym,
}

\setbeamertemplate{itemize item}{\scriptsize$\blacktriangleright$}
\setbeamertemplate{itemize subitem}{\scriptsize$\blacktriangleright$}

%Titel:
\title{The Muon Puzzle in Cosmic Ray Induced Air Showers}
%Autor
\author[N.Breer]{Nils Breer}
%Lehrstuhl/Fakultät
\institute{Fakultät Physik}
%Titelgrafik muss ich einfueren!!!
%\titlegraphic{\includegraphics[width=0.3\textwidth]{content/Bilder/interferenz.jpg}}
\date{17.06.2022}

\begin{document}
\maketitle

\begin{frame}\frametitle{Agenda}
  \begin{itemize}
    \item What is the Muon Puzzle?
    \begin{itemize}
      \item Cosmic rays and their behaviour with the atmosphere
      \item Air showers and their properties
      \item Muon discrepancy between simulation and experiment (8$\sigma$ offset)
      \item experimental validation through WHISP group
    \end{itemize}
    \item Muon Puzzle might suggest mismodelling QCD
    \begin{itemize}
      \item Nuclear effects important
      \item forwards hadron production studies from LHC data
    \end{itemize}
  \end{itemize}
\end{frame}

\begin{frame}\frametitle{Where do we see the Muon Puzzle?}
  \begin{itemize}
    \item Studies about high-energy cosmic rays through extensive air showers
    \item interpretation through models -> QCD under extreme conditions
    \item understanding the mass composition through $N_\mu$ observable
    \item Simulation deficit compared to measurement starting at TeV scale
  \end{itemize}
\end{frame}

\begin{frame}\frametitle{Cosmic rays}
  \begin{columns}
    \begin{column}[c]{0.48\textwidth}
      \includegraphics[width=\textwidth]{CR_cta.png}
    \end{column}
    \begin{column}[c]{0.48\textwidth}
      \textbf{Messengers of high-energy universe}
      \begin{itemize}
        \item gamma rays: many of them, straight from the source, E < 100 TeV
        \item neutrinos: straight from source, very rare but can be high energetic
        \item Cosmic Rays (CR): high energies, lots of them, path is highly random
      \end{itemize}
    \end{column}
  \end{columns}
\end{frame}

\begin{frame}\frametitle{Cosmic ray properties}
  \begin{itemize}
    \item discovered by Victor Hess in 1912 (balloon experiment)
    \item Fully ionised nuclei, from protons up to iron, negligible fractions
    to higher nuclei
    \item arriving earth with relativistic energies
    \item origin: unknown sources outside solar system
    \item shock acceleration (< 1 PeV) in SNR, higher energies have unknown
    mechanisms, extra-galactic > 1 EeV
    \item charged and scattered through inhomogenous fields -> random arrival directions
    \item E < 100 TeV: directly observed by space-based experiments (AMS-02\footnote{Alpha Magnetic Spectrometer})
    \item higher energies: flux too low -> ground based experiments
    (Auger, Telescope Array) through particle showers
  \end{itemize}
\end{frame}

\begin{frame}{Cosmic ray flux}
  \includegraphics[width=0.9\textwidth]{knee_heel.png}
\end{frame}

% \begin{frame}{Cosmic ray flux}
%   \begin{itemize}
%     \item Flux is scaled with $E^{2.6}$ -> many orders of magnitude
%     \item open sybols: shower experiments measuring "all particle CR flux"
%     \item coloured: flux of individual balloon and satelite measurements
%     \item empirical fit to the data (what is empirical?)
%     \item interesting part from above the knee at $\SI{1e6}{\giga\electronvolt}$.
%   \end{itemize}
% \end{frame}

\begin{frame}{Air shower model (Heitler-Matthews)}
  \begin{columns}
    \begin{column}[c]{0.48\textwidth}
      \includegraphics[width=\textwidth]{HM_model.png}
    \end{column}
    \begin{column}[c]{0.48\textwidth}
      \begin{itemize}
        \item shower simplified to pions
        \item charged pions decay to muons at low energies (end of cascade)
        \item neutral pions decay directly and form em-shower
      \end{itemize}
      \textbf{Most muons and neutrinos produced come from the end of the hadronic cascade}
      \begin{itemize}
        \item hadronic interactions need to be studied further
        \item soft hadronic cascades in forward direction
      \end{itemize}
    \end{column}
  \end{columns}
\end{frame}

\begin{frame}{Pierre Auger Observatory}
  \begin{figure}
    \centering
    \includegraphics[width=0.6\textwidth]{auger_both.jpg}
  \end{figure}
\end{frame}

\begin{frame}\frametitle{Pierre Auger Experiment}
  \begin{itemize}
    \item located in Argentina
    \item CR Energies observation between $\num{1e17}$ and $\num{1e20}$ eV
    \item studies particle interactions with water tanks at surface
    \item tracking air showers through UV light in atmosphere
    \item ground: duty cycle roughly 100\%
    \item fluorescence: roughly 15\% (needs to be dark)
  \end{itemize}
\end{frame}

% \begin{frame}{The Telescope Array}
%   \begin{figure}
%     \centering
%     \includegraphics[width=0.7\textwidth]{TCA.png}
%   \end{figure}
% \end{frame}

% \begin{frame}\frametitle{Telescope Array}
%   \begin{itemize}
%     \item hybrid experiment from many collaborations (US, Japan, Korea, Russia, Belgium)
%     \item observe air showers from CR at highest energies
%     \item combination of air-flourescence (atmospheric trace) and ground-based
%     \item scintillating trackers (footprint when reaching the surface)
%   \end{itemize}
% \end{frame}

\begin{frame}\frametitle{Cosmic Ray detection}
  \textbf{What is needed for a cosmic rays detection?}
  \begin{itemize}
    \item $\symbf{Energy}$ from size of the electromagnetic component
    \item Arrival $\symbf{direction}$ $\phi$, $\theta$ from the particles
    \item $\symbf{Mass}$ from depth of shower maximum and muon number
  \end{itemize}
  $X_{max} =$ depth where the number of secondary particles reaches a maximum \\
  $N_{\mu} =$ Number of muons
\end{frame}

\begin{frame}\frametitle{mean logarithmic mass prediction}
  \begin{columns}
    \begin{column}[c]{0.48\textwidth}
      \includegraphics{lnA_left.png}
    \end{column}
    \begin{column}[c]{0.48\textwidth}
      \begin{itemize}
        \item search dominant sources of CR -> for low fluxes need air showers
        \item Air showers are indirectly observed; mass composition sumarized by
        mean logarithmic mass <lnA> %for E above PeV
        \item because of the intrinsic fluctuations inside the air showers
        \item precise measurements can rule out competing theories (e.g CR with
        highest energies are light or heavy)
      \end{itemize}
    \end{column}
  \end{columns}
\end{frame}

\begin{frame}\frametitle{logarithmic mass prediction}
  \begin{columns}
    \begin{column}[c]{0.48\textwidth}
      \includegraphics{lnA_right.png}
    \end{column}
    \begin{column}[c]{0.48\textwidth}
      \begin{itemize}
        \item bands constructed from several measurements on air showers
        \item mass-sensitive features: $X_{max}$, $N_{\mu}$
        \item band width \to theoretical uncertainties (forward hadron production)
        \item uncertainties prevent exclusion of theories on the CR origin
        \item $N_{\mu}$ good discrimination between light and heavy rays at EeV scale
        \item more usefull than $X_{max}$ because of few statistics of flourescence
      \end{itemize}
    \end{column}
  \end{columns}
\end{frame}

\begin{frame}\frametitle{CR mass composition}
  \begin{columns}
    \begin{column}[c]{0.48\textwidth}
      \begin{figure}
        \includegraphics[width=0.8\textwidth]{lnA_all.png}
        \caption{Based on Kampert and Unger, Astropart. Phys. 35 (2012) 660}
      \end{figure}
    \end{column}
    \begin{column}[c]{0.48\textwidth}
      \textbf{What are the origins of cosmic rays?}
      \begin{itemize}
        \item Mass composition (<lnA>) of CR provides information about source and propagation
        \item uncertainties of <lnA> confined by uncertainty of hadronic interaction model
        \item \textbf{Muon Puzzle}: Predicted number of muons in air showers higher than in simulations
      \end{itemize}
    \end{column}
  \end{columns}
  \textbf{Possible solution from already taken data at the LHC}
  \begin{itemize}
    \item forward production cross-section of $\pi$, K, p
    \item forward measurements of $R = (E_{\pi^0}) / (E_\text{other hadrons})$ of em-cascades
  \end{itemize}
\end{frame}

% \begin{frame}\frametitle{experimental uncertainty}
%   \includegraphics[width=0.9\textwidth]{unc.png}
% \end{frame}
%
% \begin{frame}\frametitle{experimental uncertainty measurements}
%     \begin{itemize}
%       \item precise air shower measurements
%       \item experimental uncertainty is 10\%
%       \item factor 2.5 to 4 (E dependent) small than band width (theoretical unc.)
%       \item theo. unc. comes from shower simulation used to infer $ln(A)$ from $X_{max}$ and $N_{\mu}$
%       \item simulation essential: no way of calibrating since mass composition of any astrophysical source unknown
%       \item uncertainty from evolution of hadronic cascades; responsible for muon production at the end
%       \end{itemize}
% \end{frame}

\begin{frame}{Muon deficit in simulation}
  WHISP: Working group for Hadronic Interactions and Shower Physics \\
  formed by several experiments to increase significance by viewing more phase space \\
  \begin{columns}
    \begin{column}[c]{0.48\textwidth}
      \includegraphics[width=\textwidth]{z_zmass.png}
    \end{column}
    \begin{column}[c]{0.48\textwidth}
      \begin{itemize}
        \item Calibrate diverse measurements to common, abstract z-scale
        \item $z = \frac{\symup{ln}N_\mu \,-\, \symup{ln}N_{\mu,p}^\text{sim}}
        {\symup{ln}N_{\mu,FE}^\text{sim} \,-\, \symup{ln}N_{\mu,p}^\text{sim}}$
        \item to cancel potential biases, insensitive to mismodelling of $N_{\mu}$
        \item Deficit in air shower sim. visible around $\SI{8e16}{\electronvolt}$ (8 TeV)
        \item Slope is non-zero at 8$\sigma$
      \end{itemize}
    \end{column}
  \end{columns}
\end{frame}

\begin{frame}{The LHCb experiment}
  \begin{columns}
    \begin{column}[c]{0.48\textwidth}
      \includegraphics[width=\textwidth]{lhcb_side.jpg}
    \end{column}
    \begin{column}[c]{0.48\textwidth}
      \begin{itemize}
        \item fully instrumented at 2 < $\eta$ < 5 -> soft hadronic interactions
        \item good particle identification (optimal for $\mu$, p, $K^{\pm}$, $\pi^{\pm}$)
        \item very good momentum and vertex resolution
      \end{itemize}
    \end{column}
  \end{columns}
\end{frame}

\begin{frame}{Using the LHC for air showers}
  \begin{columns}
    \begin{column}[c]{0.48\textwidth}
      \includegraphics[width=\textwidth]{lhc_shower.png}
    \end{column}
    \begin{column}[c]{0.48\textwidth}
      \begin{itemize}
        \item p-O collisions similar to air shower interactions
        \item needed: pp, p-Pb, p-O for nuclear effects
      \end{itemize}
    \end{column}
  \end{columns}
\end{frame}

\begin{frame}{impact of LHC measurements}
  \begin{columns}
    \begin{column}[c]{0.48\textwidth}
      \includegraphics{x_max_impact.png}
    \end{column}
    \begin{column}[c]{0.48\textwidth}
      \begin{itemize}
        \item modified hadron multiplicity $N_\text{mult}$
        \item modified R ratio $= \frac{E_{\pi^0}}{E_\text{other hadrons}}$
        \item top left: pure iron shower, bottom right: pure proton shower
        \item grey: standard prediction from EPOS-LHC model
        \item muon discrepancy: distance between auger data and grey
        \item -> nuclear modifications for forward-produced hadrons
      \end{itemize}
    \end{column}
  \end{columns}
\end{frame}

\begin{frame}{Nuclear effects in forward production}
  \begin{columns}
    \begin{column}[c]{0.48\textwidth}
      \includegraphics[width=\textwidth]{suppression.png}
    \end{column}
    \begin{column}[c]{0.48\textwidth}
      \begin{itemize}
        \item close to 50\% suppression for forward direction
        \item strong suppression in CR pseudorapidity range
      \end{itemize}
    \end{column}
  \end{columns}
\end{frame}

\begin{frame}\frametitle{Why is solving the puzzle necessary?}
  \begin{itemize}
    \item resolve ambiguity of cosmic ray mass composition at EeV level
    \item improve hadronic interaction models for CR mass composition in simulation
    \item more precision of lepton flux, main background for IceCube
  \end{itemize}
\end{frame}

%%% hier muss siunitx eingebunden werden
% \begin{frame}\frametitle{Possible solutions to the Puzzle}
%   \begin{itemize}
%     \item use LHCb as instrumentation device because it has the correct eta range (2 to 5)
%   \end{itemize}
% \end{frame}

\begin{frame}\frametitle{Possible solutions}
  \begin{itemize}
    \item increase muon number by reducing energy fraction lost to photon production ($\pi^0$ decay)
    \item highest energy CR have heavy nuclei -> first interaction creates quark-gluon plasma -> shift equilibrium to an enhanced strangeness state (40\% more muons)
  \end{itemize}
\end{frame}

\begin{frame}\frametitle{Summary}
  \begin{itemize}
    \item muon deficit experimentally established with 8$\sigma$ evidence
    \item $\sqrt{s_{NN}} = 8$TeV: should be observable at LHC
    \item most likely explanation: modification in hadron production (photon energy fraction)
    \item ALICE observed enhanced strangeness in mid-rapidity which would match but needs to be studied further
    \item LHCb needed to perform missing measurements
  \end{itemize}
\end{frame}

\begin{frame}\frametitle{Quellen}
\url{http://www.telescopearray.org/index.php/about/telescope-array} \\
\url{https://www.researchgate.net/figure/A-schematic-of-the-Pierre-Auger-Observatory-where-each-black-dot-is-a-water-Cherenkov_fig1_319524774} \\
\url{https://www.cta-observatory.org/pevatrons-hunt-for-galactic-cosmic-rays/} \\
\url{https://arxiv.org/pdf/2105.06148.pdf} \\
\url{https://doi.org/10.1007/978-3-319-63411-1_1} \\
\url{https://www.sciencedirect.com/science/article/pii/S0927650512000382} \\
\url{https://cerncourier.com/a/lhcbs-momentous-metamorphosis/} \\
% \url{https://arxiv.org/pdf/1412.5106.pdf} \\
% \url{https://www.epj-conferences.org/articles/epjconf/pdf/2019/14/epjconf_ricap2019_01015.pdf} \\
% \url{https://pos.sissa.it/358/890/pdf} \\
\end{frame}

\end{document}

% \begin{frame}\frametitle{}
%   \begin{columns}
%   \begin{column}[c]{0.45\textwidth}
%     \begin{itemize}
%       \item
%       \item
%       \item
%       \item
%     \end{itemize}
%   \end{column}
%   \begin{column}[c]{0.45\textwidth}
%     % \includegraphics{images/icecube.png}
%   \end{column}
%   \end{columns}
% \end{frame}
