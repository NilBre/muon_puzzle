% This example is meant to be compiled with lualatex or xelatex
% The theme itself also supports pdflatex
\PassOptionsToPackage{unicode}{hyperref}
\documentclass[aspectratio=1610, 9pt]{beamer}

% Warning, if another latex run is needed
% \usepackage[aux]{rerunfilecheck}

% just list chapters and sections in the toc, not subsections or smaller
\setcounter{tocdepth}{1}

%------------------------------------------------------------------------------
%------------------------------ Fonts, Unicode, Language ----------------------
%------------------------------------------------------------------------------
\usepackage{fontspec}
\defaultfontfeatures{Ligatures=TeX}  % -- becomes en-dash etc.

% german language
\usepackage{polyglossia}
\setdefaultlanguage{german}

% for english abstract and english titles in the toc
\setotherlanguages{english}

% intelligent quotation marks, language and nesting sensitive
\usepackage[autostyle]{csquotes}

% microtypographical features, makes the text look nicer on the small scale
\usepackage{microtype}

%------------------------------------------------------------------------------
%------------------------ Math Packages and settings --------------------------
%------------------------------------------------------------------------------

\usepackage{amsmath}
\usepackage{amssymb}
\usepackage{mathtools}
\usepackage{bbold}

% Enable Unicode-Math and follow the ISO-Standards for typesetting math
\usepackage[
  math-style=ISO,
  bold-style=ISO,
  sans-style=italic,
  nabla=upright,
  partial=upright,
]{unicode-math}
\setmathfont{Latin Modern Math}

% nice, small fracs for the text with \sfrac{}{}
\usepackage{xfrac}


%------------------------------------------------------------------------------
%---------------------------- Numbers and Units -------------------------------
%------------------------------------------------------------------------------

\usepackage[
  locale=DE,
  separate-uncertainty=true,
  per-mode=symbol-or-fraction,
]{siunitx}
\sisetup{math-micro=\text{µ},text-micro=µ}
% \sisetup{tophrase={{ to }}}
%------------------------------------------------------------------------------
%-------------------------------- tables  -------------------------------------
%------------------------------------------------------------------------------

\usepackage{booktabs}       % \toprule, \midrule, \bottomrule, etc

%------------------------------------------------------------------------------
%-------------------------------- graphics -------------------------------------
%------------------------------------------------------------------------------

\usepackage{graphicx}
%\usepackage{rotating}
\usepackage{grffile}
\usepackage{tikz}
\usepackage{circuitikz}
\usepackage{tikz-feynman}
\usepackage{subcaption}

% allow figures to be placed in the running text by default:
\usepackage{scrhack}
\usepackage{float}
\floatplacement{figure}{htbp}
\floatplacement{table}{htbp}

% keep figures and tables in the section
\usepackage[section, below]{placeins}


%------------------------------------------------------------------------------
%---------------------- customize list environments ---------------------------
%------------------------------------------------------------------------------

\usepackage{enumitem}
\usepackage{listings}
\usepackage{hepunits}

\usepackage{pdfpages}
%------------------------------------------------------------------------------
%------------------------------ Bibliographie ---------------------------------
%------------------------------------------------------------------------------

\usepackage[
  backend=biber,   % use modern biber backend
  autolang=hyphen, % load hyphenation rules for if language of bibentry is not
                   % german, has to be loaded with \setotherlanguages
                   % in the references.bib use langid={en} for english sources
]{biblatex}
\addbibresource{references.bib}  % the bib file to use
\DefineBibliographyStrings{german}{andothers = {{et\,al\adddot}}}  % replace u.a. with et al.


% Load packages you need here
% \usepackage{polyglossia}
% \setmainlanguage{german}

\usepackage{csquotes}


% \usepackage{amsmath}
% \usepackage{amssymb}
% \usepackage{mathtools}

\usepackage{hyperref}
\usepackage{bookmark}

% load the theme after all packages

\usetheme[
  showtotalframes, % show total number of frames in the footline
]{tudo}

% Put settings here, like
\unimathsetup{
  math-style=ISO,
  bold-style=ISO,
  nabla=upright,
  partial=upright,
  mathrm=sym,
}

%Titel:
\title{The Muon Puzzle}
%Autor
\author[N.Breer]{Nils Breer}
%Lehrstuhl/Fakultät
\institute{Fakultät Physik}
%Titelgrafik muss ich einfueren!!!
%\titlegraphic{\includegraphics[width=0.3\textwidth]{content/Bilder/interferenz.jpg}}
\date{17.06.2022}

\begin{document}
\maketitle

\begin{frame}\frametitle{Agenda}
  \begin{itemize}
    \item What is the Muon Puzzle?
    \item Why do we want to study it?
    \item Cosmic rays and their behaviour with the atmosphere
    \item air shower: trivia and properties
    \item How do we measure these phenomena and which experiments are used?
    \item Other problems related to the muon puzzle
    \item possible solutions
  \end{itemize}
\end{frame}

\begin{frame}\frametitle{The Muon Puzzle}
  \begin{itemize}
    \item indirect observation of cosmic rays through air showers in atmosphere
    \item interpretation -> accurate models of air shower physics (QCD extreme)
    \item air showers -> hadronic cascades rich in muons
    \item $N_\mu$ key observable for mass composition of CR
    \item Simulation shows drastic Muon deficit compared to measurement! -> why?
    \item visible at TeV scale -> LHC also didn't observe that!
  \end{itemize}
\end{frame}

\begin{frame}\frametitle{Why is solving the puzzle interesting}
  \begin{itemize}
    \item reduce the size of $N_\mu$ bands by a factor of 2.5 to 4
    \item resolve ambiguity (mehrdeutigkeit) of cosmic ray mass composition at EeV level
    \item improve hadronic interaction models for CR mass composition in simulation
    \item more precision of lepton flux, main background for IceCube
  \end{itemize}
\end{frame}

\begin{frame}\frametitle{What are cosmic rays?}
  \begin{itemize}
    \item discovered by Victor Hess in 1912 (balloon experiment)
    \item Fully ionised nuclei, from protons up to iron, negligible fractions
    to higher nuclei
    \item arriving earth with relativistic energies
    \item come from unknown sources outside the solar system
    \item shock acceleration (< 1 PeV) in SNR, higher energies have unknown
    mechanisms, extra-galactic > 1 EeV
  \end{itemize}
\end{frame}

\begin{frame}\frametitle{More about cosmic rays}
  \begin{itemize}
    \item CR may come from point-like sources, don't appear as such -> isotropic flux
    \item charged and scattered through inhomogenous fields
    \item E < 100 TeV: directly observed by space-based experiments (AMS-02\footnote{Alpha Magnetic Spectrometer})
    \item higher energies: flux too low -> ground based experiments
    (Auger, Telescope Array) through particle showers
  \end{itemize}
\end{frame}

\begin{frame}\frametitle{Pierre Auger Experiment}
  \begin{itemize}
    \item
    \item
    \item
  \end{itemize}
\end{frame}

\begin{frame}\frametitle{Telescope Array}
  \begin{itemize}
    \item
    \item
    \item
  \end{itemize}
\end{frame}

\begin{frame}\frametitle{}
  \begin{itemize}
    \item
    \item
    \item
  \end{itemize}
\end{frame}

\begin{frame}\frametitle{}
  \begin{itemize}
    \item Grundlage ist das studieren von kosmischer Strahlen durch Luftschauer
    in der Atmosphäre
    \item Aus Distanz- und Energiegr\"unden werden QCD Modelle unter extremen
    Bedingungen auf die Probe gestellt
    \item Luftschauer in der Atmosphäre bilden hadronische Kaskaden die Muonen
    im Endzustand aufweisen
    \item Hauptsächlich 10 bis 100 GeV Muonen
    \item Simulation zeigt starkes defizit in Muonen gg\"u. gemessenen auf
    \item Diskrepanzen nicht erklärbar
    \item Phänomen schon bei 8 TeV Luftschauer beobachtbar, auch potenziell am LHC
    \item also warum passiert das?
  \end{itemize}
\end{frame}

\begin{frame}\frametitle{Was sind kosmische Strahlen}
  \begin{itemize}
    \item Ionisierte kerne, hochenergetische geladene Teilchen mit relativistischen
    Energien
    \item von Protonen bis zu Eisen
    \item unbekannte extragalaktische Herkunft ausserhalb des Sonnensystems
    \item Differentieller Fluss J(E) proportional zu dN / dE circa $E^\alpha$
    mit $\alpha = 2.6$ um die enorme Skala abzubilden
    \item Energieskala 11 Größenordnungen, Fluss 30 Größenordnungen
    \item Viele Experimente versuchen die Skala vollständig zu beschreiben
    \item Diffuser Fluss der kosmischen Strahlen durch das Universum wegen
    inhomogener Felder
    \item
  \end{itemize}
\end{frame}

\begin{frame}\frametitle{}
  \begin{itemize}
    \item a
    \item b
    \item c
  \end{itemize}
\end{frame}

\begin{frame}\frametitle{Muon Messungen und Modelle}
  \begin{itemize}
    \item d
  \end{itemize}
\end{frame}

\begin{frame}\frametitle{M\"ogliche L\"osungen f\"ur das Puzzle}
  \begin{itemize}
    \item e
  \end{itemize}
\end{frame}

%%% hier muss siunitx eingebunden werden

\begin{frame}\frametitle{Zusammenfassung}
  \begin{itemize}
    \item Muon Defizit klar erkennbar in Luftschauern mit 8\sigma
    \item Wichtige Beitr\"age von IceCube und Auger an modell-abh\"angigen Messungen
    \item $\sqrt{S_{NN}} \approx \SI{8}{\tera\electronvolt}$ mit linearem Anstieg in $\symup{log}(E)$ -> hoch energie messungen am LHC
    \item Naheliegendste Erkl\"arung: kleine modifikation in hadron produktion welches der energiebeitrag der photonen reduziert. meist aus $\pi^{0}$ Zerf\"allen
    \item kleine modifikationen haben nahezu keinen Einfluss in heutigen Luftschauer simulationen deswegen unentdeckt
    \item erh\"ohte strageness produktion wurde von ALICE in der mid-rapidit\"ats region gemessen
  \end{itemize}
\end{frame}

\begin{frame}\frametitle{Quellen}
\url{https://icecube.wisc.edu/science/icecube/detector} \\
\url{https://micro.magnet.fsu.edu/primer/digitalimaging/concepts/photomultipliers.html} \\
\url{http://antares.in2p3.fr/} \\
\url{https://ecap.nat.fau.de/index.php/research/neutrino-astronomy/antares-km3net/} \\
\url{http://www.km3net.org/} \\
\url{https://baikalgvd.jinr.ru/} \\
\url{https://masterok.livejournal.com/2364208.html} \\
\url{https://arxiv.org/pdf/1412.5106.pdf} \\
\url{https://www.epj-conferences.org/articles/epjconf/pdf/2019/14/epjconf_ricap2019_01015.pdf} \\
\url{https://pos.sissa.it/358/890/pdf} \\
\end{frame}

\end{document}

% \begin{frame}\frametitle{}
%   \begin{columns}
%   \begin{column}[c]{0.45\textwidth}
%     \begin{itemize}
%       \item
%       \item
%       \item
%       \item
%     \end{itemize}
%   \end{column}
%   \begin{column}[c]{0.45\textwidth}
%     % \includegraphics{images/icecube.png}
%   \end{column}
%   \end{columns}
% \end{frame}
